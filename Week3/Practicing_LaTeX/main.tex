% Preamble (import packages)
\documentclass[12pt]{article}
\usepackage[utf8]{inputenc}
\usepackage[margin=1in]{geometry}  % sets the margins to 1 inch every side
\usepackage{amssymb}
\usepackage{cite}
\usepackage{apacite}
\usepackage{graphicx}

% Title header
\title{Cats are cool!}
\author{Aneesh}
\date{}

\begin{document}

\maketitle % This just displays the title we created above on the page

\section{Introduction}
\subsection{Alignment}
% \textbf = bold a word or sentence
% \textit = italicize word or sentence
% \underline = underline word or sentence
This is the \underline{\textit{\textbf{alignment}}} section!!! \\

% This part shows how to center text
\underline{\textbf{Center}}
\begin{center}
Example 1: The following paragraph (given in quotes) is an 
example of Center Alignment using the center environment. 

``LaTeX is a document preparation system and document markup 
language. LaTeX uses the TeX typesetting program for formatting 
its output, and is itself written in the TeX macro language. 
LaTeX is not the name of a particular editing program, but 
refers to the encoding or tagging conventions that are used 
in LaTeX documents".
\end{center}


% This part shows how to flush left
\underline{\textbf{Flush Left}}
\begin{flushleft}
Example 1: The following paragraph (given in quotes) is an 
example of Center Alignment using the center environment. 

``LaTeX is a document preparation system and document markup 
language. LaTeX uses the TeX typesetting program for formatting 
its output, and is itself written in the TeX macro language. 
LaTeX is not the name of a particular editing program, but 
refers to the encoding or tagging conventions that are used 
in LaTeX documents".
\end{flushleft}


% This part shows how to flush right
\underline{\textbf{Flush Right}}
\begin{flushright}
Example 1: The following paragraph (given in quotes) is an 
example of Center Alignment using the center environment. 

``LaTeX is a document preparation system and document markup 
language. LaTeX uses the TeX typesetting program for formatting 
its output, and is itself written in the TeX macro language. 
LaTeX is not the name of a particular editing program, but 
refers to the encoding or tagging conventions that are used 
in LaTeX documents".
\end{flushright}


\underline{\textbf{Begin Paragraphs}} \\
This is the text in first paragraph. This is the text in first 
paragraph. This is the text in first paragraph. \par
This is the text in second paragraph. This is the text in second 
paragraph. This is the text in second paragraph.


\newpage


\section{Section 1: Writing Mathematics}
% Equation of a line
\begin{equation}
    y = mx + b
\end{equation}

% The Quadratic equation
\begin{equation}
    f(x) = ax^2 + bx + c
\end{equation}

% Quadratic formula
To find the roots of a quadratic equation, we can use the following formula:
\begin{equation}
    x = \frac{-b \pm \sqrt{b^2 - 4ac}}{2a}
\end{equation}

% Greek symbols
Often, in deep learning, we set the learning rate, $\alpha$, to $0.01$. \\

\begin{center}
    Beta: $\beta$ \\
Gamma: $\gamma$
\end{center}

We define a function $f: \mathbb{R} \rightarrow \mathbb{R}$. \\

\begin{equation}
    \sum_{i=0}^5 i = 15
\end{equation}

\begin{equation}
    \prod_{i=1}^5 i = 5! = 120
\end{equation}


\newpage

\section{Section 2: Bibliographies}
There has been a lot of improvements in Multi-Choice Machine Reading Comprehension (MMRC) including reformulations of the problem as Single-Choice to allow the use of Transfer Learning to improve existing models \cite{jiang2020improving}.



\newpage

\section{Section 3: Images}

% Inserting an image and captioning it
\begin{figure}[!h]
    \centering
    \includegraphics[scale=0.2]{AEOP_Logo.jpeg}
    \caption{This is an image of the AEOP logo}
    \label{fig:my_label}
\end{figure}



\section{Section 4: Tables}

\begin{center}
 \begin{tabular}{c c} % each c specifies a column (in this case we want 2 columns) 
 \hline
 Models & Test \\ [0.5ex] 
 \hline  % Horizontal lines
 Roberta (Liu et al., 2019) & 83.2  \\ 
 ALBERT + DUMA (ensemble) (Zhu et al., 2020) & 89.8  \\
 ALBERT + DUMA (ensemble) (Zhu et al., 2020) & 89.8  \\
 ALBERT + DUMA (ensemble) (Zhu et al., 2020) & 89.8  \\
 ALBERT + DUMA (ensemble) (Zhu et al., 2020) & 89.8  \\
 ALBERT + DUMA (ensemble) (Zhu et al., 2020) & 89.8  \\
 ALBERT + DUMA (ensemble) (Zhu et al., 2020) & 89.8  \\ [1ex] 
 \hline
 ALBERT + DUMA (ensemble) (Zhu et al., 2020) & 89.8  \\ 
 ALBERT + DUMA (ensemble) (Zhu et al., 2020) & 89.8  \\
 ALBERT + DUMA (ensemble) (Zhu et al., 2020) & 89.8  \\
 ALBERT + DUMA (ensemble) (Zhu et al., 2020) & 89.8 \\
 ALBERT + DUMA (ensemble) (Zhu et al., 2020) & 89.8  \\
 ALBERT + DUMA (ensemble) (Zhu et al., 2020) & 89.8  \\ [1ex] 
 \hline
\end{tabular}
\end{center}




\newpage

\section{Conclusion}


\bibliographystyle{apacite}

\bibliography{bibfile}

\end{document}
